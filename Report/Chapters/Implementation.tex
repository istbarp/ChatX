\chapter{Implementation}

The implementation section below will explain what technologies were used and how they were implemented into ChatX.

\section{Code Standards}
Since this project makes use of two different programming languages it would be unwise to not set a code standard. So a code standard was set for each programming language.

\subsection{Java}
Class names must start with a capital letter.
Variables must start with lowercase letters, private ones should be refereed by using ".this".
Curly brackets are to be in their own lines at all times.
Method names must start with a lowercase letter.


\subsection{C\#}
Class names must start a capital letter.
Variables must start with lowercase letters, private ones should be refereed by using ".this".
Curly brackets are to be in their own lines at all times.
Method names must start with a capital letter.

\section{Client}
The "Client" is what the person at home will use to use the chat system. There will be two different clients to cater to as many as possible.

\subsection{Java Client}

The java client has been written as the first client and will be used as the test client as it is possible to have a console application.

\subsection{Web Client}

The web client will be what caters to most people since it's a thing that can be used on all platforms, such as phones, pc's, and even consoles. It will be the most difficult to make as well since it requires AJAX (Update a website without reloading).

\section{Server}

\subsection{Webservice}

\section{RabbitMQ}