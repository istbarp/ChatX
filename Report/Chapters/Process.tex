\chapter{Process}
This part of the report will cover the process used whilst developing ChatX. A agile method was adopted but a few practices that were outside of the chosen method also have seen some use, such as Test First and Pair Programming.

\section{Kanban}
Due to time pressure and considering how agile Kanban is, it was deemed the best methodology to follow for ChatX. Kanban has very few rules and encourages changing flow of process if it helps a development team to deliver more value to the software, it also puts emphasis on how important visualization of work flow is therefore a virtual kanban board (figure \ref{fig:KanbanBoard})  was made on Kanban-flow.com and used. 

\begin{figure}[H]
\centering
\includegraphics[width=0.7\linewidth]{"img/Kanban Board"}
\caption{Kanban board}
\label{fig:KanbanBoard}
\end{figure}

The Kanban board is used to help keep an overview of what assignments need to get done, Kanban boards aren't never strict and can always be changed to suit a projects needs.
This specific Kanban Board has five coloumns.To-do, Do Today, In progress, To verify and Done. The To-do column is used to keep track of the assignments that are left to be completed in order to complete the entire project. The next column is Do Today, assignments from To-do are picked and then put over to Today. This helps keep track of how fast the workflow is in the team and whether it can handle more assignments each day or needs to take a step back and do fewer. In progress column shows which assignments are being currently worked on, it helps to show which assignments are already taken so that code work won't be duplicated and not wasting working hours. Once a assignment is completed it is then being put over to the To Verify column where other team members have to take a look at the results of the assignment and verify it's been made properly. Once an assignment has been verified it is then moved to the final column which is named Done, it shows all assignments that have been completed.

\section{Reflection}