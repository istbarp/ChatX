\chapter{Process}
This part of the report will cover the process used whilst developing ChatX. An agile method was adopted, and Pair Programming was added here for harder tasks.

\section{Kanban}
Due to time pressure and considering how agile Kanban is, it was deemed the best methodology to follow for ChatX. Kanban has few rules and encourages changing flow of process if it helps a development team to deliver more value to the software. It also puts emphasis on how important visualization of work flow is. Figure \ref{fig:KanbanBoard} shows an example of a the Kanban board used.  %FOOTNOTE: on Kanbanflow.com and used. 

\begin{figure}[H]
	\centering
	\includegraphics[width=0.7\linewidth]{"img/Kanban Board"}
	\caption{Kanban board}
	\label{fig:KanbanBoard}
\end{figure}

This Kanban Board has five columns: To-do, Do Today, In Progress, To verify and Done. 
The To-do column on the Kanban Board is used to keep track of the tasks that are left to be completed in order to complete the entire project. The next column is Do Today, tasks from To-do are picked and then put over to Today. This helps keep track of how fast the work flow is in the team and whether it can handle more tasks each day or needs to take a step back and do fewer. The In Progress column shows which tasks are being currently worked on, it helps to show which tasks are already taken so that code work will not be duplicated and hours wasted. Once a task is completed it is then put over to the To Verify column, where other team members have to take a look at the results of the task and verify it has been made properly. Once a task has been verified, it is then moved to the final column which is named Done. This column shows all tasks which have been completed. 

\section{Reflection}

Unfortunately, the Kanban rules and board that were set have barely been followed, due to time pressure and how frequently the system design was changed before a final design was decided on. In retrospect it would have been better if a methodology was decided upon earlier rather than as late as it happened in this project and putting more effort into following the rules. Instead of communicating through the Kanban board, ideas and tasks were handled by direct communication.

\section{Code Standards}
Since this project makes use of two different programming languages it would be unwise to not set a code standard. So a code standard was set for each programming language.

\subsection{Java}
Class names must start with a capital letter.
Variables must start with lowercase letters, private ones should be refereed by using the "this" keyword.
Curly brackets are to be on their own lines at all times.
Method names must start with a lowercase letter.


\subsection{C\#}
Class names must start a capital letter.
Variables must start with lowercase letters, private ones should be refereed by using the "this" keyword.
Curly brackets are to be on their own lines at all times.
Method names must start with a capital letter.




