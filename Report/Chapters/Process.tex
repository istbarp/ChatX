\chapter{Process}
This chapter will cover the process used whilst developing ChatX. An agile method was adopted but a few practices that were outside of the chosen method also have seen some use, such as Test First and Pair Programming.

\section{Kanban}
Due to time pressure and considering how agile Kanban is, it was deemed the best methodology to follow for ChatX. Kanban has few rules and encourages changing flow of process if it helps a development team to deliver more value to the software, it also puts emphasis on how important visualization of work flow is therefore a virtual Kanban board (figure \ref{fig:KanbanBoard}) was made.  %FOOTNOTE: on Kanbanflow.com and used. 

\begin{figure}[H]
\centering
\includegraphics[width=0.7\linewidth]{"img/Kanban Board"}
\caption{Kanban board}
\label{fig:KanbanBoard}
\end{figure}

%The Kanban board is used to help keep an overview of what assignments need to get done, Kanban boards are never strict and can always be changed to suit a projects needs. // needed?
This Kanban Board has five columns. To-do, Do Today, In progress, To verify and Done. 
The To-do column on the Kanban Board is used to keep track of the tasks that are left to be completed in order to complete the entire project. The next column is Do Today, tasks from To-do are picked and then put over to Today. This helps keep track of how fast the work flow is in the team and whether it can handle more tasks each day or needs to take a step back and do fewer. In progress column shows which assignments are being currently worked on, it helps to show which tasks are already taken so that code work won't be duplicated and not wasting working hours. Once a task is completed it is then being put over to the To Verify column where other team members have to take a look at the results of the task and verify it's been made properly. Once a task has been verified it is then moved to the final column which is named Done, it shows all assignments that have been completed. %"Assignment" changed to "task"

\section{Reflection}

Unfortunately, the methodology rules that were set have barely been followed. Due to time pressure and size of the project the kanban was not used, and instead assignments and ideas were communicated through direct communication. It was also difficult to follow the methodology since the design of the system changed a lot and frequently, the final draft of the design ended up being very different from the original one. In retrospect it would have been better if the methodology was decided on earlier, and more effort into trying to follow it, even if the project isn't that big.